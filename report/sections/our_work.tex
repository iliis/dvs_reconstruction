The goal of this project is to implement full rotational tracking and scene
reconstruction using only a DVS. We implemented an approach described in Kim et
al. in \cite{kim2014simultaneous} using Matlab. It uses a SLAM-like two-step
algorithm, where the position and the map are jointly estimated and is
described in Section \ref{sec:core_algorithm}.

In contrast to \cite{kim2014simultaneous} we're working with some
simplifications: The particle filter for the tracking directly usese Euler
angles and corner cases like gimbal lock are not handled.

To speed up things up, we assume that the camera's initial field
of view is already known and included in the map. With this assumption we do
not have to deal with a special initialization procedure and can immediately
start with the standard iteration while being relatively sure that the camera
movement is tracked correctly.

This is not an unreasonable assumption, as newer DVS models incorporate the
ability to take full-frame pictures and there's a simple method for generating
full images which works with any model (see Section \ref{sec:shutter_removal}).





Explain basic algo, more details in Section 5: 'core algorithm'.


-some simplifications\\
    -use particle filter directly with euler angles (ignore corner cases)\\
    -start with first fov in map\\





// TODO - insert section to explain what we did here

general goal, how to achieve this

-implemented paper (kim) in matlab\\

