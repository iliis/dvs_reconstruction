The goal of this project was to implement full rotational tracking and scene
reconstruction using only a DVS. We implemented an approach described by Kim \etal
in \cite{kim2014simultaneous} using Matlab. It uses a SLAM-like two-step
algorithm, where the position and the map are jointly estimated and is
described in Section \ref{sec:core_algorithm}.

In contrast to \cite{kim2014simultaneous} we are working with some
simplifications: The particle filter for the tracking directly uses Euler
angles and corner cases like gimbal lock are not handled.

As we discovered during the implementation, the runtime of the algorithm is
extremely long when implemented in Matlab. The events that are created by the
camera during only a few seconds could easily lead to hours of computation.
To still be able to test the algorithm we exclusively used data generated with
our simulation instead of the real camera.

This enabled us to control the number of events by tweaking parameters of our
virtual camera.

A further benefit of this approach was the existence of an exact ground truth
for the camera motion which greatly simplified the evaluation of the algorithm.

To speed things up, we assume that the camera's initial field
of view is already known and included in the map. With this assumption we do
not have to deal with a special initialization procedure and can immediately
start with the standard iteration while being relatively sure that the camera
movement is tracked correctly.

This is not an unreasonable assumption, as newer DVS models incorporate the
ability to take full-frame pictures and there's a simple method for generating
full images which works with any model (see Section \ref{sec:shutter_removal}).
