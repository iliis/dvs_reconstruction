In order to facilitate development we implmemented a full DVS simulator that
generates events from a spherical panorama.  This allows to develop and test
each part of the algorithm independently with perfect input data and ground
truth. It also makes experimentation easier as everything can be inspected at
will.

The simulation starts off with a normal image containing a 360\textdegree
panorama. The field of view of the camera is computed given its orientation
(normal Euler angles) and the panorama image (the ground truth map) is sampled
using raycasting. The camera is then rotated by a tiny amount to get a second
image patch. These patches are then subtracted to get a list of events where
the difference exceeds a threshold.

This discrete approach is fairly straightforward but suffers from the problem
that it is very slow, as very tiny steps have to be taken to get good events.

\begin{figure}
\includegraphics[width=\linewidth]{images/simulation_raw.jpg}
\includegraphics[width=\linewidth]{images/simulation_events.jpg}
\caption{above: the map used as input with the camera's FOV, below: events generated by moving the camera to the right}
\label{fig:simulation}
\end{figure}
