A Dynamic Vision Sensor (DVS) \cite{lpd08dvs}, also known as event camera, is a new kind of camera. In contrast to a standard camera id does not output a complete image of the scene, but a series of events. Hereby, an event is generated when the sensed brightness of a pixel changes more than a certain threshold. This happens for each pixel completely independently from all others.

A DVS has several advantages compared to standard cameras. Since there is no need to gather data from all pixels to generate an image, events can be send with an extremely low latency of 15 $\mu$s or less \cite{lpd08dvs, bandli14davis}. For the same reason, there is practically no motion blur in the DVS signal. Another benefit of independent pixels is the very high dynamic range. Blooming or smearing as observed when a bright light source is captured with a standard camera is basically nonexistent. Finally, due to the reduction of the camera signal to changes and the resulting neglect of redundant information, the signal bandwith is much smaller than with a standard camera.
All these features make DVS a very desirable sensor for motion tracking, especially for mobile robots where the data has to be analyzed on a constraint hardware.

There are, however, several downsides at the current point in time. First of all, currently available DVS, (DVS128 \cite{lpd08dvs}, DAVIS \cite{brandli14davis}) habe a very limited resolution of 128 $\times$ 128 or 240 $\times$ 180 pixels, respectively. This obviously limits the spacial accuracy of the sensed data, but will likely only be a matter of time until sensors with a higher resolution are developed. A more important point is that the computer vision algorithms that have been developed during the last decades can either not be used at all or have to be adapted to the new data representation.